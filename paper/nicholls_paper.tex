
%% bare_conf.tex
%% V1.4b
%% 2015/08/26
%% by Michael Shell
%% See:
%% http://www.michaelshell.org/
%% for current contact information.
%%
%% This is a skeleton file demonstrating the use of IEEEtran.cls
%% (requires IEEEtran.cls version 1.8b or later) with an IEEE
%% conference paper.
%%
%% Support sites:
%% http://www.michaelshell.org/tex/ieeetran/
%% http://www.ctan.org/pkg/ieeetran
%% and
%% http://www.ieee.org/

%%*************************************************************************
%% Legal Notice:
%% This code is offered as-is without any warranty either expressed or
%% implied; without even the implied warranty of MERCHANTABILITY or
%% FITNESS FOR A PARTICULAR PURPOSE! 
%% User assumes all risk.
%% In no event shall the IEEE or any contributor to this code be liable for
%% any damages or losses, including, but not limited to, incidental,
%% consequential, or any other damages, resulting from the use or misuse
%% of any information contained here.
%%
%% All comments are the opinions of their respective authors and are not
%% necessarily endorsed by the IEEE.
%%
%% This work is distributed under the LaTeX Project Public License (LPPL)
%% ( http://www.latex-project.org/ ) version 1.3, and may be freely used,
%% distributed and modified. A copy of the LPPL, version 1.3, is included
%% in the base LaTeX documentation of all distributions of LaTeX released
%% 2003/12/01 or later.
%% Retain all contribution notices and credits.
%% ** Modified files should be clearly indicated as such, including  **
%% ** renaming them and changing author support contact information. **
%%*************************************************************************


% *** Authors should verify (and, if needed, correct) their LaTeX system  ***
% *** with the testflow diagnostic prior to trusting their LaTeX platform ***
% *** with production work. The IEEE's font choices and paper sizes can   ***
% *** trigger bugs that do not appear when using other class files.       ***                          ***
% The testflow support page is at:
% http://www.michaelshell.org/tex/testflow/



\documentclass[conference]{IEEEtran}
% Some Computer Society conferences also require the compsoc mode option,
% but others use the standard conference format.
%
% If IEEEtran.cls has not been installed into the LaTeX system files,
% manually specify the path to it like:
% \documentclass[conference]{../sty/IEEEtran}





% Some very useful LaTeX packages include:
% (uncomment the ones you want to load)


% *** MISC UTILITY PACKAGES ***
%
%\usepackage{ifpdf}
% Heiko Oberdiek's ifpdf.sty is very useful if you need conditional
% compilation based on whether the output is pdf or dvi.
% usage:
% \ifpdf
%   % pdf code
% \else
%   % dvi code
% \fi
% The latest version of ifpdf.sty can be obtained from:
% http://www.ctan.org/pkg/ifpdf
% Also, note that IEEEtran.cls V1.7 and later provides a builtin
% \ifCLASSINFOpdf conditional that works the same way.
% When switching from latex to pdflatex and vice-versa, the compiler may
% have to be run twice to clear warning/error messages.






% *** CITATION PACKAGES ***
%
%\usepackage{cite}
% cite.sty was written by Donald Arseneau
% V1.6 and later of IEEEtran pre-defines the format of the cite.sty package
% \cite{} output to follow that of the IEEE. Loading the cite package will
% result in citation numbers being automatically sorted and properly
% "compressed/ranged". e.g., [1], [9], [2], [7], [5], [6] without using
% cite.sty will become [1], [2], [5]--[7], [9] using cite.sty. cite.sty's
% \cite will automatically add leading space, if needed. Use cite.sty's
% noadjust option (cite.sty V3.8 and later) if you want to turn this off
% such as if a citation ever needs to be enclosed in parenthesis.
% cite.sty is already installed on most LaTeX systems. Be sure and use
% version 5.0 (2009-03-20) and later if using hyperref.sty.
% The latest version can be obtained at:
% http://www.ctan.org/pkg/cite
% The documentation is contained in the cite.sty file itself.






% *** GRAPHICS RELATED PACKAGES ***
%
\ifCLASSINFOpdf
   \usepackage[pdftex]{graphicx}
  % declare the path(s) where your graphic files are
  % \graphicspath{{../pdf/}{../jpeg/}}
  % and their extensions so you won't have to specify these with
  % every instance of \includegraphics
  % \DeclareGraphicsExtensions{.pdf,.jpeg,.png}
\else
  % or other class option (dvipsone, dvipdf, if not using dvips). graphicx
  % will default to the driver specified in the system graphics.cfg if no
  % driver is specified.
  % \usepackage[dvips]{graphicx}
  % declare the path(s) where your graphic files are
  % \graphicspath{{../eps/}}
  % and their extensions so you won't have to specify these with
  % every instance of \includegraphics
  % \DeclareGraphicsExtensions{.eps}
\fi
% graphicx was written by David Carlisle and Sebastian Rahtz. It is
% required if you want graphics, photos, etc. graphicx.sty is already
% installed on most LaTeX systems. The latest version and documentation
% can be obtained at: 
% http://www.ctan.org/pkg/graphicx
% Another good source of documentation is "Using Imported Graphics in
% LaTeX2e" by Keith Reckdahl which can be found at:
% http://www.ctan.org/pkg/epslatex
%
% latex, and pdflatex in dvi mode, support graphics in encapsulated
% postscript (.eps) format. pdflatex in pdf mode supports graphics
% in .pdf, .jpeg, .png and .mps (metapost) formats. Users should ensure
% that all non-photo figures use a vector format (.eps, .pdf, .mps) and
% not a bitmapped formats (.jpeg, .png). The IEEE frowns on bitmapped formats
% which can result in "jaggedy"/blurry rendering of lines and letters as
% well as large increases in file sizes.
%
% You can find documentation about the pdfTeX application at:
% http://www.tug.org/applications/pdftex





% *** MATH PACKAGES ***
%
%\usepackage{amsmath}
% A popular package from the American Mathematical Society that provides
% many useful and powerful commands for dealing with mathematics.
%
% Note that the amsmath package sets \interdisplaylinepenalty to 10000
% thus preventing page breaks from occurring within multiline equations. Use:
%\interdisplaylinepenalty=2500
% after loading amsmath to restore such page breaks as IEEEtran.cls normally
% does. amsmath.sty is already installed on most LaTeX systems. The latest
% version and documentation can be obtained at:
% http://www.ctan.org/pkg/amsmath





% *** SPECIALIZED LIST PACKAGES ***
%
%\usepackage{algorithmic}
% algorithmic.sty was written by Peter Williams and Rogerio Brito.
% This package provides an algorithmic environment fo describing algorithms.
% You can use the algorithmic environment in-text or within a figure
% environment to provide for a floating algorithm. Do NOT use the algorithm
% floating environment provided by algorithm.sty (by the same authors) or
% algorithm2e.sty (by Christophe Fiorio) as the IEEE does not use dedicated
% algorithm float types and packages that provide these will not provide
% correct IEEE style captions. The latest version and documentation of
% algorithmic.sty can be obtained at:
% http://www.ctan.org/pkg/algorithms
% Also of interest may be the (relatively newer and more customizable)
% algorithmicx.sty package by Szasz Janos:
% http://www.ctan.org/pkg/algorithmicx




% *** ALIGNMENT PACKAGES ***
%
%\usepackage{array}
% Frank Mittelbach's and David Carlisle's array.sty patches and improves
% the standard LaTeX2e array and tabular environments to provide better
% appearance and additional user controls. As the default LaTeX2e table
% generation code is lacking to the point of almost being broken with
% respect to the quality of the end results, all users are strongly
% advised to use an enhanced (at the very least that provided by array.sty)
% set of table tools. array.sty is already installed on most systems. The
% latest version and documentation can be obtained at:
% http://www.ctan.org/pkg/array


% IEEEtran contains the IEEEeqnarray family of commands that can be used to
% generate multiline equations as well as matrices, tables, etc., of high
% quality.




% *** SUBFIGURE PACKAGES ***
%\ifCLASSOPTIONcompsoc
%  \usepackage[caption=false,font=normalsize,labelfont=sf,textfont=sf]{subfig}
%\else
  \usepackage[caption=false,font=footnotesize]{subfig}
%\fi
% subfig.sty, written by Steven Douglas Cochran, is the modern replacement
% for subfigure.sty, the latter of which is no longer maintained and is
% incompatible with some LaTeX packages including fixltx2e. However,
% subfig.sty requires and automatically loads Axel Sommerfeldt's caption.sty
% which will override IEEEtran.cls' handling of captions and this will result
% in non-IEEE style figure/table captions. To prevent this problem, be sure
% and invoke subfig.sty's "caption=false" package option (available since
% subfig.sty version 1.3, 2005/06/28) as this is will preserve IEEEtran.cls
% handling of captions.
% Note that the Computer Society format requires a larger sans serif font
% than the serif footnote size font used in traditional IEEE formatting
% and thus the need to invoke different subfig.sty package options depending
% on whether compsoc mode has been enabled.
%
% The latest version and documentation of subfig.sty can be obtained at:
% http://www.ctan.org/pkg/subfig




% *** FLOAT PACKAGES ***
%
%\usepackage{fixltx2e}
% fixltx2e, the successor to the earlier fix2col.sty, was written by
% Frank Mittelbach and David Carlisle. This package corrects a few problems
% in the LaTeX2e kernel, the most notable of which is that in current
% LaTeX2e releases, the ordering of single and double column floats is not
% guaranteed to be preserved. Thus, an unpatched LaTeX2e can allow a
% single column figure to be placed prior to an earlier double column
% figure.
% Be aware that LaTeX2e kernels dated 2015 and later have fixltx2e.sty's
% corrections already built into the system in which case a warning will
% be issued if an attempt is made to load fixltx2e.sty as it is no longer
% needed.
% The latest version and documentation can be found at:
% http://www.ctan.org/pkg/fixltx2e


%\usepackage{stfloats}
% stfloats.sty was written by Sigitas Tolusis. This package gives LaTeX2e
% the ability to do double column floats at the bottom of the page as well
% as the top. (e.g., "\begin{figure*}[!b]" is not normally possible in
% LaTeX2e). It also provides a command:
%\fnbelowfloat
% to enable the placement of footnotes below bottom floats (the standard
% LaTeX2e kernel puts them above bottom floats). This is an invasive package
% which rewrites many portions of the LaTeX2e float routines. It may not work
% with other packages that modify the LaTeX2e float routines. The latest
% version and documentation can be obtained at:
% http://www.ctan.org/pkg/stfloats
% Do not use the stfloats baselinefloat ability as the IEEE does not allow
% \baselineskip to stretch. Authors submitting work to the IEEE should note
% that the IEEE rarely uses double column equations and that authors should try
% to avoid such use. Do not be tempted to use the cuted.sty or midfloat.sty
% packages (also by Sigitas Tolusis) as the IEEE does not format its papers in
% such ways.
% Do not attempt to use stfloats with fixltx2e as they are incompatible.
% Instead, use Morten Hogholm'a dblfloatfix which combines the features
% of both fixltx2e and stfloats:
%
% \usepackage{dblfloatfix}
% The latest version can be found at:
% http://www.ctan.org/pkg/dblfloatfix




% *** PDF, URL AND HYPERLINK PACKAGES ***
%
%\usepackage{url}
% url.sty was written by Donald Arseneau. It provides better support for
% handling and breaking URLs. url.sty is already installed on most LaTeX
% systems. The latest version and documentation can be obtained at:
% http://www.ctan.org/pkg/url
% Basically, \url{my_url_here}.




% *** Do not adjust lengths that control margins, column widths, etc. ***
% *** Do not use packages that alter fonts (such as pslatex).         ***
% There should be no need to do such things with IEEEtran.cls V1.6 and later.
% (Unless specifically asked to do so by the journal or conference you plan
% to submit to, of course. )


% correct bad hyphenation here
\hyphenation{op-tical net-works semi-conduc-tor}


\begin{document}
%
% paper title
% Titles are generally capitalized except for words such as a, an, and, as,
% at, but, by, for, in, nor, of, on, or, the, to and up, which are usually
% not capitalized unless they are the first or last word of the title.
% Linebreaks \\ can be used within to get better formatting as desired.
% Do not put math or special symbols in the title.
\title{NYC Green Taxis with Neural Network}


% author names and affiliations
% use a multiple column layout for up to three different
% affiliations
\author{\IEEEauthorblockN{Brittany Nicholls}
\IEEEauthorblockA{Department of Computer Science\\ and Electrical Engineering\\
University of Maryland, Baltimore County\\
Baltimore, Maryland 21250\\
Email: brn1@umbc.edu}}

% conference papers do not typically use \thanks and this command
% is locked out in conference mode. If really needed, such as for
% the acknowledgment of grants, issue a \IEEEoverridecommandlockouts
% after \documentclass

% for over three affiliations, or if they all won't fit within the width
% of the page, use this alternative format:
% 
%\author{\IEEEauthorblockN{Michael Shell\IEEEauthorrefmark{1},
%Homer Simpson\IEEEauthorrefmark{2},
%James Kirk\IEEEauthorrefmark{3}, 
%Montgomery Scott\IEEEauthorrefmark{3} and
%Eldon Tyrell\IEEEauthorrefmark{4}}
%\IEEEauthorblockA{\IEEEauthorrefmark{1}School of Electrical and Computer Engineering\\
%Georgia Institute of Technology,
%Atlanta, Georgia 30332--0250\\ Email: see http://www.michaelshell.org/contact.html}
%\IEEEauthorblockA{\IEEEauthorrefmark{2}Twentieth Century Fox, Springfield, USA\\
%Email: homer@thesimpsons.com}
%\IEEEauthorblockA{\IEEEauthorrefmark{3}Starfleet Academy, San Francisco, California 96678-2391\\
%Telephone: (800) 555--1212, Fax: (888) 555--1212}
%\IEEEauthorblockA{\IEEEauthorrefmark{4}Tyrell Inc., 123 Replicant Street, Los Angeles, California 90210--4321}}




% use for special paper notices
%\IEEEspecialpapernotice{(Invited Paper)}




% make the title area
\maketitle

% As a general rule, do not put math, special symbols or citations
% in the abstract
\begin{abstract}
We propose a methodology for determining which of Baltimore's neighborhoods a new resident would very likely want to avoid.
Specifically, focus is placed on using the datasets provided by official local government agencies through Open Baltimore
Beta to accomplish this goal.  Additionally, we ensure our approach to solving this problem is scalable so that it could support ``big data"
amounts of data.

The basic idea behind our approach to determining which neighborhoods a new resident would need to be wary of is to 1)
generate features based off of a new inhabitants concerns 2) determine which neighborhoods are similar to each other based
on the features and then 3) configure which neighborhoods are least desirable without requiring human intervention.  Using
this methodology, we successfully identify 20 neighborhoods in Baltimore City that new residents should avoid.

\end{abstract}

% no keywords




% For peer review papers, you can put extra information on the cover
% page as needed:
% \ifCLASSOPTIONpeerreview
% \begin{center} \bfseries EDICS Category: 3-BBND \end{center}
% \fi
%
% For peerreview papers, this IEEEtran command inserts a page break and
% creates the second title. It will be ignored for other modes.
\IEEEpeerreviewmaketitle



\section{Introduction}
These days, when people think of Baltimore City, many individuals refer to The Wire, a TV show created by HBO that ran from 2002-2008.

addresses a small subsection of Baltimore's society.
Furthermore, in the past few years, Baltimore has received national attention for events such as the death of Freddie

Freddie Gray's death at the forefront of society's recent memory, people attribute a negative reputation to the entirety of Baltimore.  While it is not a ``politcally correct" thing to say, Baltimore is like any big city.  There are certain neighborhoods in Baltimore that are perfectly enjoyable and safe 
and at the same time there exist other parts of the city that most people will want to avoid living in unless they are familiar with the neighborhood.  

Unfortunately, people who are moving to Baltimore do not have any easy, reliable resources that will recommend neighborhoods they might want
to avoid until they understand different neighborhoods in the city operate.  If someone starts googling to try to find this information, they will
end up on one of three subpar resources: 1) Forums, 2) Maps that contain crime statistics or 3) News articles, such as


A city should not be avoided simply because it ranks high in a list of the ``most dangerous" cities in the United States;
it just means one should be conscious of what part of the city one lives in.  The goal of this paper is to use Open Source Data to determine
which neighborhoods a new resident should avoid living in, while making sure that the methodology can scale well to support large amounts
of data.  As more government agencies and cities are releasing data, this would become a ``big data" problem,
and the current approach would be able to be extended to other cities.


\subsection{Motivation}
The motivation for this paper is inspired by both of the authors experience with moving to Baltimore.  Both were aware that
Baltimore is like any other big city, in which there are parts one should avoid living in, but looking online
does not provide any obvious guidance and they just wanted one easy resource that does more than look at crime.

Currently, resources available to people looking to decide where to move
in Baltimore are to visit a bunch of different neighborhoods, to read forums where opinions wildly differ from one another and can be confusing, to look at
crime statistics provided by online tools, or to try to piece together what neighborhoods news articles are talking about.
Additionally, new residents have multiple questions they are asking as they search for a place to live that go beyond
the crime in the neighborhood.

The list of questions this approach would help new residents answer are as follow:
\begin{enumerate}
	\item What neighborhoods are relatively safe?
	\item What neighborhoods are people actually living in?
	\item What neighborhoods are clean?
	\item What neighborhoods are being invested in?
\end{enumerate}

\section{Related Work}
\subsection{Crime Reports}
Looking at the data produced by Uniform Crime Reporting Program for the FBI,

the only cities where there were more murders, and compared to Baltimore's population of 620K,
they have a population of 2.7 million and 8.5 million people,  only does Baltimore
have a high number of murders, it also has a lower population.
As can be seen with the FBI report, reading crime stats can be intimidating and do not paint the ``full picture" of the city.  Thus,
Baltimore should not be avoided just because it has more crime in certain areas than other cities do.  This paper is
trying to identify neighborhoods that residents who are unfamiliar with the city might want to avoid, whereas this crime
report is looking at the city as a whole.

\subsection{Using Machine Learning to Understand the Impact of Mixing on Neighborhoods}
Very recently, machine learning was applied to understand how the impact of mixing people with different socioeconomic
backgrounds into one neighborhood would affect the neighborhood as aplied a new technique,
kernel regularized least squares, to census data from between 2000 and 2010 in order to determine how the change in a neighborhoods
characteristics affected the average income in that neighborhood.  While this is an example of someone applying machine learning
techniques to neighborhood data, they are looking to solve a very different problem.  However, potential future work of
this paper would include analyzing the neighborhoods over time in order to build a better recommendation system, and the
approach proposed by Hipp, Kbe useful in this future work.

\subsection{Baltimore Neighborhood Indicators Alliance}
For the past 15 years, the Baltimore Neighborhood Indicators Alliance, of the Jacob Frances Institute, has released a
Vital Signs report that relies on hard data about Baltimore’s Community Statistical Areas (CSA) in order to assess the quality
of life in a CSA and work towards im-FJI Vital Signs are intended to be used in
order to track the quality of life over time as well as provide hard data as valuable input into decision making when planning new
programs and the future of t

While the BNIA-FJI work provides valuable insight into how a community is doing, the large areas that the CSAs cover do
not allow for the more detailed approach

%\begin{figure*}[h!]
%\centering
%\subfloat[Zip Code]{\includegraphics[width=.3\textwidth]{zip_neighborhood.png}%
%\label{zip}
%}
%\hfil
%\subfloat[Census Tracts]{\includegraphics[width=.3\textwidth]{census_neighborhood.png}%
%\label{census}
%}
%\hfil
%\subfloat[Community Statistical Areas]{\includegraphics[width=.3\textwidth]{csa_neighborhood.png}%
%\label{csa}
%}
%\caption{Possible geographies for Baltimore City as presented by BNIA-FJI \cite{bniaGeo}}
%\label{fig_sim}
%\end{figure*}

\subsection{Unofficial Neighborhood Analysis}
There are almost no published results of researchers analyzing neighborhoods; however, Ken Steif and others are doing
urban spatial analysis and publishing informal white papers based on their ata.
While predicting future home prices would be useful for a new resident who is looking to buy a house in Baltimore, this is too specific for
our scope as we are trying to build a recommendation for any new resident, whether they are renting or buying.  Again,
this could fall into future work.

\section{Data and Data Challenges}
All of the data was pulled from official sources off of the Open Baltimore Beta 

\subsection{Victim Based Crime Dataset}
The Baltimore Police Department posted an official dng Part 1 Victim Based Crimes from January 1, 2012
until April 8, 2017. The data stops on April 8, 2017 because that is the day of the last update of the data before the
dataset was pulled down to be analyzed.

The crimes included in this dataset include:
% Make this a table? Explain the crimes?
\begin{itemize}
	\item Aggravated Assault
	\item Arson
	\item Assault by Threat
	\item Auto Theft
	\item Burglary
	\item Common Assault
	\item Homicide
	\item Larceny
	\item Larceny from Auto
	\item Rape
	\item Carjacking (Robbery)
	\item Commercial Robbery
	\item Residential Robbery
	\item Street Robbery
	\item Shooting
\end{itemize}

The other information included in this dataset are approximate address, geocode, neighborhood, date and time, as well as
a few details on whether the crime was indoors or outdoors and if a weapon was involved.
The original dataset contains approximately 255K rows, which were filtered down to about 242K rows of data that were
clean and unique.

It is important to note there are some limitations in this data due to legal concerns. The addresses provided in the
dataset are approximate estimations of the true location of the crime and addresses that could not be geocoded were
not included in the original dataset.


\subsection{ECB Citations Dataset}
The ECB Citations Dataset is provided by the Baltimore City Environmental Con Department,
Department of Transportation, Department of General Services, Department of Public Works, Department of Housing and
Community Development, and the Department of Recreation and Parks.

The citations are for various activities, violations, and problems that are not necessarily criminal and have to do with
the community and common good.  The violations range from trash accumulation and littering to issues with a building and
yard, or even hunting/fishing violations.  The dataset includes information such as the fine amount, balance amount,
date of violation, agency writing the violation and violation description, as well as general information about the
location of the violation.


\subsection{Property Tax Dataset}
Real Estate Property Tax information was provided by the Baltimore City Department of Find in the dataset.
Fortunately, most of the data was pretty clean and included information such as the amount due in city and state tax,
whether or not the property was a principal residence, as well as general information about the property.  Most of the
data that was filtered out was due to there being no tax information, but upon manual analysis it appears that most of those properties are
either parks or seem to be vacant or condemned.

One thing to note is that a property was deemed either as a Principal Residence or Not a Principal Residence.  A property
is considered a principal residence if the property is owner occupied. So buildings that are not a principal residence
include rentals as well as commercial property.


\subsection{Vacant Buildings}
A danformation contained in this dataset include the address of the vacancy, the date the vacancy notice was given,
in addition to other information about the location of the vacancy.

This dataset was fairly clean and straightforward.  One thing to note is that the date the vacancy notice was given is not necessarily the date the property was vacant.
In one instance, looking at Google's Street View history for a property showed the property had been boarded up and
was vacant years before the notice was given. For example, the house at 802 N Castle Street has as vacant date as of June 5, 2013; however, looking at
Street View it is clear the house had not been lived in since November 2007 as it was boarded up.


\subsection{Housing Permits}
Baltimore Housing's Office of Permits and Building Inspections posted a dataset comprised of the submitted permits for
both residential and commercial prhe permits are for things as simple as replacing a water heater, or for as
difficult as constructing new medical buildings.

The data contains information such as the address of the property for the permit, a description and cost estimate of the work to be done,
as well as general information about the location and permit.  Approximately 370K permits were analyzed for the purposes
of this paper.


\subsection{Challenge of Working with Data}
Since all of the data was provided through Open Baltimore Beta, it was fairly clean as it is meant for public consumption.
One of the biggest problems encountered was data where some rows had important columns, such as the address, missing.
As long as less than 10\% of the records were missing valuable information, the dataset was used.

An additional problem faced is that the datasets were not thoroughly described on Open Baltimore Beta, although all of the
datasets had easy to understand column names. In order to gain a basic understanding of what the datasets contained, each
one had to be explored individually, with some additional help from outside resources in order to clarify some details in
the data.

One of the datasets that really had to be cleaned up was the Part 1 Victim Based Crime dataset.  In it, there were
 quite a few records that were duplicated. This was assumed to mean that multiple people were involved.  For instance, there
 were about 50 duplicate 

\subsection{Other Datasets Considered}
The Open Baltimore Beta iniative made other datasets available that could have been used in determining which neighborhoods
new residents may not want to live in.
Unfortunately, the following datasets were available, but were not used due to easy to find inaccuracies such as missing locations:

\subsubsection{Grocery Stores}
A potential list of grocery stores, including national-chain supermarkets as well as smaller local convenience stores.

\subsubsection{Homeless Shelters}
A potential list of homeless shelters, both emergency and transitional.

\section {Challenge of Determining Geography Method}

There are a few different approaches that could be taken in order to deal with the geography of a city.  The census breaks
Baltimore city into 200 traddition, the BNIA-JFI previous research breaks the city
into Community Statistical Areas which are clusters of census tr
for the outlines of the CSAs.  The reasoning for relying on the census tracts
is the ability to measure the progress of an area over tunfamiliar with Baltimore's neighborhoods would prefer to avoid.  Thus, the
changes that occur in a constant section of a city do not matter for the purposes of this paper and thus, we did not continue
with the use of CSAs.

An additional way to break a city in smaller chunks would be to look at zipcodes. Faccording to the population movement.

Thus, when working with new residents, breaking a city down into neighborhoods is the ideal approach .  Neighborhoods are somewhat
fluid and capture the changes that are occurring within the city.  Neighborhoods change as their occupants move and as
the buildings change.  In addition, people naturally break a city down into neighborhoods, so it is a very intuitive way
of breaking down a city.  Lastly, neighborhoods are much smaller sections of the city, but large enough to capture trends especially
as people who are similar tend to live near each other.


\section{Identifying Neighborhoods to Avoid}
By far, the largest challenge is finding hard evidence or research pointing to specific neighborhoods as places a person
might want to avoid.  It is not ``politically correct" to point out which neighborhoods most people would want to avoid;
however, it is common knowledge that every big city has areas that are not the best to live in if it can be avoided, especially
if you are not familiar with the neighborhood and how it operates.  One of the primary reasons why we avoid creating features
based on demographics is so that the focus remains on objective features that would make a neighborhood ``good" or ``bad" to live in.

As pointed out in the Motivation section, forums reflect the opinions of locals in that a lot of people do not agree about
a neighborhoods livability based on their own tolerances and preconceptions.
Fortunately, there are a handful of neighborhoods that are notorious for not being the easiest to live in.  These are the
neighborhoods that this paper is trying to determine.

For instance, the neighborhood of Sandtown-Winchester has received a lot of attention as the place where Freddie Gray lived
and was arrhermore, a Washington Post article also points
out that Sandtown-Winchester is affected by its history with crack and is still s

A Baltimore Sun article discussed the issues within the neighborhood of Coldstream Homestead Montebello, aka the Chum.
In the Chum, there are a lot of gangs who have more or less spread the message of "no snitchi
Interestingly, this article also pointed out one of the positive effects that a project that could be the future work
 of this paper could help with:

\begin{quote}
``But many residents believe gun violence defines the city more than it should, pointing to multibillion-dollar waterfront
developments, national attractions and major league sports teams...police to politicians, know the high homicide rate
threatens economic vitality and efforts to draw new residents. And they are scrambling to stop
\end{quote}

With this in mind, our approach is considered successful if it identifies the neighborhoods of Coldstream Homestead Montello,
Harlem Park, and Sandtown-Winchester, as well as neighborhoods that are similar to these neighborhoods,
 as neighborhoods that should be avoided by new residents.

\section {Naive Approach}
The first reasonable approach one might take in trying to determine which neighborhoods to avoid living in would be to look at
the victim based crime levels in the neighborhood.y the Baltimore Police Departm red.

%\begin{table}
%\centering
%    \begin{tabular}{| l | l |}
%    \hline
%    \textbf{Neighborhood} & \textbf{Crime Count} \\ \hline
%    Downtown & 8029 \\ \hline
%    Frankford & 5875  \\ \hline
%    Belair-Edison & 5353  \\ \hline
%    Brooklyn & 3807  \\ \hline
%    Sandtown-Winchester & 3527  \\
%    \hline
%    \end{tabular}
%    \caption{Top 5 Neighborhoods with the Highest Victim-Based Crime Counts}
%    \label{naive}
%\end{table}

While it would make sense that someone would want to live in a low crime area, crime does not paint the whole picture about
the livability of a neighborhood.  Many factors affect crime counts.
In the case of Downtown, there are a lot of tourists, so crime is naturally higher.  Frankford, Belair-Edison and Brooklyn are
among the top populated neighborhoods in Baltimore, so it makes sense that with more people there is more crimes.

%\begin{figure}
%  \centering
%    \includegraphics[width=\linewidth]{bad_baltimore_map.png}
%      \caption{Baltimore City with the High Count Crime Neighborhoods Highlighted}
%      \label{badBmore}
%\end{figure}

\section{Determining Livability}
As shown previously, looking purely at crime is not the best way of determining which neighborhoods should be avoided as there are many
reasons why crime may be higher in one neighborhood than it is in another.  Since this method of determining which neighborhoods someone should live in is targeting new
residents looking to move to Baltimore City, it needs to take into account multiple factors that would affect whether or
not someone should live in a neighborhood.  This method is also agnostic of cost of living and other factors that may be
specific to a user.

Very generally, features are generated using the datasets discussed above.  Apache Spark 1.6.2 is used to
manipulate the data from Baltimore is small enough to be worked with on a laptop.
After the features are generated, they are run through the K-Means clustering algorithm that is predefined in Spark
in order to determine which neighborhoods have similar features.  At this point, each neighborhood will be assigned to
a cluster along with other neighborhoods with similar features.  To determine which cluster represents the place where a new resident would not want
to live, the average of the scores for each feature for each cluster is summed up and the cluster with the highest score
is ranked as the least desirable neighborhood.

\subsection{Feature Selection}
As mentioned in the Motivation section, this approach is trying to answer the following questions:
\begin{enumerate}
	\item What neighborhoods are relatively safe?
	\item What neighborhoods are people actually living in?
	\item What neighborhoods are being invested in?
	\item What neighborhoods are clean?
\end{enumerate}

Therefore, the features were created to allow neighborhoods that look similar to each other based on how they would answer
the above questions to be clustered together.

Note that all of the below features are normalized from 0 to 100 to ensure that each feature has equal weight.

To clarify, active properties are all properties that are currently paying tax, regardless of whether they are owner occupied
or not.

\subsubsection{Exceptional Violence Crime Count}
For each neighborhood, the number of victim based crimes that have been classified as a shooting or homicide are counted.
Of the victim based crimes, shootings and homicides are are indicative of exceptional violence and potentially
fatal, so they are deserving of their own feature.  The number of exceptional violence crimes are not normalized against the population
or active properties because this is the type of crime that most residents do not want to have in their neighborhood, regardless
of how large the population is.

\subsubsection{Resident-Centric Crime to Property Count Ratio}
A potential resident would care about crimes such as burglary, auto theft, residential robbery, shootings and homicide
because those could directly affect them and are crimes that are hard to deter.  The count of the number of these crimes
is normalized against the number of active properties because these crimes are more likely to happen
in neighborhoods where there are more houses.

\subsubsection{Vacancy Count to Total Property Count Ratio}
A high number of vacancies in a neighborhood is very likely indicative of people leaving the neighborhood and properties
not being properly maintained.  Once again, the number of vacancies in a neighborhood is normalized against the number
of active properties because some neighborhoods have more properties than others.

\subsubsection{ECB Citation Count to Property Count Ratio}
The number of ECB citations issued are indicative of how dirty and regulation-abiding the residents are.  The
number of citations is normalized against the number of active properties in order to take into account property counts in a neighborhood.
Neighborhoods with more properties will have more chances for a violation.

\subsubsection{Total Residential Property Tax to Total Non-Residential Property Tax Ratio}
In order to calculate if a neighborhood has a large amount of rentals/commercial properties or is primarily owner occupied
the ratio of the owner occupied residential taxes to the number of non-owner-occupied properties is used as a feature.
This helps group neighborhoods together that are full of owner-occupied properties, or are highly commercial, or are full
of rentals.

\subsubsection{Permit Cost to Property Count Ratio}
The ratio of the total estimated cost from permits to the total number of properties helps identify neighborhoods where
large amounts of money are being invested into it.  The total permit cost to active properties ratio helps normalize for neighborhoods
where large amounts of work are being done against many properties.

\subsection{Algorithm}
%\begin{figure}
%  \centering
%    \includegraphics[width=\linewidth]{baltimore_map.png}
%      \caption{Baltimore City with the recommended neighborhoods to avoid highlighted in red.  Neighborhoods that score
%      most positively in the features are highlight in green.}
%      \label{finalMap}
%\end{figure}

\subsubsection{Determine Neighborhoods that are Similar}
Use k-means in order to group together neighborhoods that are similar to each based on the features explained previously.
With this dataset, 5 clusters provided the ideal groupings of neighborhoods.  Tathe cluster
index, the number of neighborhoods in that cluster, as well as the average of the clusters' neighborhoods score for the respective
feature.

Note that:
\begin{itemize}
	\item EVC=Exceptional Violence Crime Count
	\item RD=Residential Crime
    \item V=Vacancies/Properties
    \item ECB=ECB Citations/Properties
    \item Tax=Residential Tax/Non-Residential Tax
    \item PC=Permit Cost/Properties
\end{itemize}
Based off of the average of the features of the
neighborhoods that were grouped together, the 5 resulting clusters separated neighborhoods into:

\begin{table}
\centering
    \begin{tabular}{| l | l | l | l | l | l | l | l |}
    \hline
    \textbf{Cluster} & \textbf{Count} & \textbf{EVC} & \textbf{RD} & \textbf{V} & \textbf{ECB} & \textbf{Tax} & \textbf{PC}\\ \hline
    1 & 62 & L & M & L & L & L & H \\ \hline
    2 & 82 & L & L & L & L & H & L \\ \hline
    3 & 75 & L & L & L & M & M/H & L \\ \hline
    4 & 43 & M & L & M & M & M & L \\ \hline
    5 & 20 & H & L & H & H & M & L \\
    \hline
    \end{tabular}
    \caption{Average of the features of the neighborhoods in the clusters}
    \label{table:clusters}
\end{table}

Lookin
are neighborhoods that are heavily commercial or are filled with rentals and a lot of work is being done on the neighborhood.
In addition, the crime is relatively low and there are not a lot of ECB citations, so those neighborhoods are clean.  Neighborhoods in this
cluster include Downtown and Inner Harbor and a lot of the parks or industrial areas, which makes sense.

Cluster 2 groups together neighborhoods that are highly residential and pretty clean and safe.  This includes neighborhoods
such as Canton, Blythewood and Ednor Gardens-Lakeside that are relatively middle-upper class.

Neighborhoods in cluster 3 are a lot like those in cluster 2 in that there are a lot of owner-occupied properties, but there
is a bit more crime and issues with ecb violations.  These neighborhoods are more working class or up-and-coming and are fairly safe.  For example,
some of the neighborhoods in this cluster are Woodmere, Belair-Edison, and Hampden.

Cluster 4 neighborhoods have more violence and crime in them than the previous groups of neighborhoods and are neighborhoods
that might want to be avoided at night.  There are also quite
a few vacancies in these neighborhoods and not as many active properties are owner occupied.  Examples of neighborhoods in this cluster
are Hollins Market, Mosher, and Easterwood.

Cluster 5 is made up of neighborhoods that new residents should avoid unless they know what they are doing.  These neighborhoods
make up a good portion of Baltimore's homicides and shootings.  The residential crime is low, but that does not mean a whole
lot since the number of vacancies in these neighborhoods are high.  These neighborhoods are also in disrepair, given the
high amount of ECB violations and low number of estimated costs from permits.  Neighborhoods in this cluster include
Sandtown-Winchester, Coldstream Homestead Montebello, Harlem Park, Mondawmin, and Broadway East.

\subsubsection{Filter for the Neighborhoods to Avoid}
While K-Means works well for grouping neighborhoods together, it does not build any sort of recommendation for which neighborhoods
a new resident should avoid if possible.

In order to determine which neighborhoods to be wary of, there needs to be a way to measure which neighborhoods do not
meet the positive requirements listed earlier.  To do this, take the average of each feature for all of the neighborhoods
in a cluster.  Then, score each cluster based on the sum of the averages of the features; the cluster with the highest sum will be the neighborhood
that a new resident would like living in the least.  Note that this method only works if a higher score for a feature correlates to
being a negative factor for a new resident.  For instance, a neighborhood with a high amount of exceptional violence crime
should have a feature score closer to 1, because people do not want to live in an area with a lot of exceptional violence crime.
In order to meet these requirements, the averages of the scores for some features had to be inverted
by subtracting the calculated average from 100 (which is the highest value from the normalization).

\subsubsection{Results}

By following the methodology described, the clusters described in Tab
\begin{enumerate}
	\item Cluster 5, whose neighborhoods are colored red in Fig
	\item Cluster 4, orange
	\item Cluster 1, yellow
	\item Cluster 3, yellow-green
	\item Cluster 2, green
\end{enumerate}

Thus, the number 1 group of neighborhoods to avoid are in cluster 5.  The 20 neighborhoods in cluster 5 are:
\begin{itemize}
	\item Boyd-Booth
	\item Broadway East
    \item Carrollton Ridge
    \item Central Park Heights
    \item Coldstream Homestead Montebello
    \item Druid Heights
    \item East Baltimore Midway
    \item Franklin Square
    \item Harlem Park
    \item Johnston Square
    \item Middle East
    \item Midtown-Edmondson
    \item Mondawmin
    \item Oliver
    \item Penn North
    \item Penrose/Fayette Street Outreach
    \item Poppleton
    \item Sandtown-Winchester
    \item Shipley Hill
    \item Upton
\end{itemize}

The final results were successful in identifying Coldstream Homestead Montebello, Harlem Park, and Sandtown-Winchester as
neighborhoods new residents should try to avoid.
Upon analyzing the rest of the neighborhoods, they have similar features to the 3 that were trying to be grouped together
in that the exceptionally violent crime is high, the vacancy rate is high, the number of ecb violations is somewhat high,
and they have lower amounts of property taxes being paid on
owner occupied properties.  These are all features that are indicative of a neighborhood that is not in the best shape and
that someone should move to if they are familiar with the neighborhood.


\subsubsection{Scalability for Big Data}
Apache Spark is a tool that can handle big data in a distributed fashion.  Because the algorithm was implemented using
the built in K-Means, that section is perfectly scalable.  In addition, the averaging of the features of the clustered neighborhoods
fits perfectly well with the Map Reduce paradigm that Apache Spark relies.  In the end, it would not be unreasonable
to collect all 5 of the final cluster scores back to the driver in order to perform a local sort and map the cluster to
the appropriate fill color, then broadcast the cluster-color mapping to the executors in order to label the neighborhoods
with their appropriate color.  All of the information could then be saved off to, say, Elasticsearch, at which point one
could modify the neighborhood outline geojson file so that the neighborhoods get filled with their appropriate color.

Simply put, this algorithm would be more than capable of handling big data should this methodology be used on open
source data for other cities.

\section{Future Work}
Possibilities for future work are discussed in this section.
\subsection{Use Technique on Other Cities}
Explore whether or not these features and this techniques translates well to other cities.
 \subsection{More Features}
 New residents have other concerns beyond what was discussed in this paper.  Parents might want to live in a good school
 district and young adults might want to live near restaurants and bars.  Adding the ability to not only identify which
 neighborhoods should be avoided, but to actually recommend neighborhoods would an interesting problem to explore.
 Additionally, importing other datasets would allow for a more diverse and larger set of features.
 However, both of these additions would require acquiring and analyzing more data, which is a long process.
 \subsection{Analysis of Neighborhoods Over Time}
 Many of the datasets had over a years worth of data.  Some future work could look at examining the neighborhoods over
 the past few years in order to identify neighborhoods that are becoming a more desirable place to live, or are becoming
 a less desirable place to live.
 \subsection{Use Technique for Other Use Cases}
 The primary context for this approach was to identify neighborhoods a new resident would not want to live in.  Could
 the same technique be used, but change to a different context.  For example, could neighborhoods be identified that
 are about to have a large population change due to deaths of the elderly or births.

% An example of a floating figure using the graphicx package.
% Note that \label must occur AFTER (or within) \caption.
% For figures, \caption should occur after the \includegraphics.
% Note that IEEEtran v1.7 and later has special internal code that
% is designed to preserve the operation of \label within \caption
% even when the captionsoff option is in effect. However, because
% of issues like this, it may be the safest practice to put all your
% \label just after \caption rather than within \caption{}.
%
% Reminder: the "draftcls" or "draftclsnofoot", not "draft", class
% option should be used if it is desired that the figures are to be
% displayed while in draft mode.
%
%\begin{figure}[!t]
%\centering
%\includegraphics[width=2.5in]{myfigure}
% where an .eps filename suffix will be assumed under latex, 
% and a .pdf suffix will be assumed for pdflatex; or what has been declared
% via \DeclareGraphicsExtensions.
%\caption{Simulation results for the network.}
%\label{fig_sim}
%\end{figure}

% Note that the IEEE typically puts floats only at the top, even when this
% results in a large percentage of a column being occupied by floats.


% An example of a double column floating figure using two subfigures.
% (The subfig.sty package must be loaded for this to work.)
% The subfigure \label commands are set within each subfloat command,
% and the \label for the overall figure must come after \caption.
% \hfil is used as a separator to get equal spacing.
% Watch out that the combined width of all the subfigures on a 
% line do not exceed the text width or a line break will occur.
%
%\begin{figure*}[!t]
%\centering
%\subfloat[Case I]{\includegraphics[width=2.5in]{box}%
%\label{fig_first_case}}
%\hfil
%\subfloat[Case II]{\includegraphics[width=2.5in]{box}%
%\label{fig_second_case}}
%\caption{Simulation results for the network.}
%\label{fig_sim}
%\end{figure*}
%
% Note that often IEEE papers with subfigures do not employ subfigure
% captions (using the optional argument to \subfloat[]), but instead will
% reference/describe all of them (a), (b), etc., within the main caption.
% Be aware that for subfig.sty to generate the (a), (b), etc., subfigure
% labels, the optional argument to \subfloat must be present. If a
% subcaption is not desired, just leave its contents blank,
% e.g., \subfloat[].


% An example of a floating table. Note that, for IEEE style tables, the
% \caption command should come BEFORE the table and, given that table
% captions serve much like titles, are usually capitalized except for words
% such as a, an, and, as, at, but, by, for, in, nor, of, on, or, the, to
% and up, which are usually not capitalized unless they are the first or
% last word of the caption. Table text will default to \footnotesize as
% the IEEE normally uses this smaller font for tables.
% The \label must come after \caption as always.
%
%\begin{table}[!t]
%% increase table row spacing, adjust to taste
%\renewcommand{\arraystretch}{1.3}
% if using array.sty, it might be a good idea to tweak the value of
% \extrarowheight as needed to properly center the text within the cells
%\caption{An Example of a Table}
%\label{table_example}
%\centering
%% Some packages, such as MDW tools, offer better commands for making tables
%% than the plain LaTeX2e tabular which is used here.
%\begin{tabular}{|c||c|}
%\hline
%One & Two\\
%\hline
%Three & Four\\
%\hline
%\end{tabular}
%\end{table}


% Note that the IEEE does not put floats in the very first column
% - or typically anywhere on the first page for that matter. Also,
% in-text middle ("here") positioning is typically not used, but it
% is allowed and encouraged for Computer Society conferences (but
% not Computer Society journals). Most IEEE journals/conferences use
% top floats exclusively. 
% Note that, LaTeX2e, unlike IEEE journals/conferences, places
% footnotes above bottom floats. This can be corrected via the
% \fnbelowfloat command of the stfloats package.




\section{Conclusion}
In this paper, we introduce a technique that will identify neighborhoods in Baltimore City that a new resident would want
to avoid living in. We successfully identified a group of neighborhoods similar to Coldstream Homestead Montebello, Sandtown-Winchester,
and Harlem Park, which were our targeted neighborhoods as they are not desirable neighborhoods to live in at the moment.
Furthermore, we succeeded in choosing an approach that is easily scalable so that it could handle big data as more and more
cities publish open source datasets that could be used in applications like this one.
Ultimately, we generated a map of Baltimore, as seen in that a new resident can easily identify which
neighborhoods they want to shy away from while looking for somewhere to live.


% conference papers do not normally have an appendix


% use section* for acknowledgment





% trigger a \newpage just before the given reference
% number - used to balance the columns on the last page
% adjust value as needed - may need to be readjusted if
% the document is modified later
%\IEEEtriggeratref{8}
% The "triggered" command can be changed if desired:
%\IEEEtriggercmd{\enlargethispage{-5in}}

% references section

% can use a bibliography generated by BibTeX as a .bbl file
% BibTeX documentation can be easily obtained at:
% http://mirror.ctan.org/biblio/bibtex/contrib/doc/
% The IEEEtran BibTeX style support page is at:
% http://www.michaelshell.org/tex/ieeetran/bibtex/
\bibliographystyle{IEEEtran}
% argument is your BibTeX string definitions and bibliography database(s)
\bibliography{IEEEabrv }%,baltimoreBib}
%
% <OR> manually copy in the resultant .bbl file
% set second argument of \begin to the number of references
% (used to reserve space for the reference number labels box)

%\bibliography{baltimoreBib}
%\bibstyle{filename}
%\nocite{*}
%\begin{thebibliography}{1}
%
%\bibitem{IEEEhowto:kopka}
%H.~Kopka and P.~W. Daly, \emph{A Guide to \LaTeX}, 3rd~ed.\hskip 1em plus
%  0.5em minus 0.4em\relax Harlow, England: Addison-Wesley, 1999.
%
%\end{thebibliography}




% that's all folks
\end{document}

